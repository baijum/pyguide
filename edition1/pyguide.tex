\documentclass[11pt, a4paper]{book}
\usepackage[
    paperwidth=8in,
    paperheight=10in,
    top=.875in,
    bottom=.875in,
    left=.875in,
    right=.875in
    ]{geometry}

\def\myauthor{Baiju Muthukadan}
\def\mytitle{A Comprehensive Guide to Python Programming}
\def\mycopyright{\myauthor}
\def\mykeywords{}
\def\mybibliostyle{plain}
\def\mybibliocommand{}
\def\mysubtitle{}
\def\myaffiliation{}
\def\myaddress{}
\def\myemail{baiju.m.mail@gmail.com}
\def\myweb{http://muthukadan.net}
\def\myphone{+91 9945973441}
\def\myversion{}
\def\myrevision{}

\date{}
\def\mykeywords{Baiju, Muthukadan, Python, Programming, Guide, Book, Tutorial}

\renewcommand\contentsname{Table Of Contents}

% donot indent first line of paragraph.
\setlength{\parindent}{0pt}
\setlength{\parskip}{5pt plus 2pt minus 1pt}

\newlength{\admonitionwidth}
\setlength{\admonitionwidth}{0.9\textwidth}
\newlength{\docinfowidth}
\setlength{\docinfowidth}{0.9\textwidth}

% sloppy
% ------
% Less strict (opposite to default fussy) space size between words. Therefore
% less hyphenation.
\sloppy

\usepackage{listings}
\usepackage[usenames,dvipsnames]{color}
\usepackage{makeidx}
\makeindex
\usepackage{tabularx}

\definecolor{listinggray}{gray}{0.9}
\lstset{language=python}
\lstset{commentstyle=\color{MidnightBlue},
        stringstyle=\color{OliveGreen},
        identifierstyle=\color{black},
        keywordstyle=\color{Bittersweet},
        showspaces=false}
%\lstset{frame=trBL,frameround=tttt}
%\lstset{backgroundcolor=\color{listinggray},rulecolor=\color{blue}}

\lstset{linewidth=\textwidth}
\lstset{ %
  basicstyle=\ttfamily\scriptsize, % the size of the fonts that are used for the code
  showstringspaces=false, % underline spaces within strings
  numbers=left, % where to put the line-numbers
  numberstyle=\scriptsize, % the size of the fonts that are used for the line-numbers
  stepnumber=1, % the step between two line-numbers. If it's 1 each line will be numbered
  numbersep=5pt, % how far the line-numbers are from the code
  numberblanklines=false,
}


\usepackage{fancyhdr}
\setlength{\headheight}{15.2pt}
\pagestyle{fancy}

\lhead[A Comprehensive Guide to Python Programming]{A Comprehensive Guide to Python Programming}
%\chead[<even output>]{<odd output>}
\rhead{}

\usepackage{fontspec}
\usepackage{xunicode}
\defaultfontfeatures{Mapping=tex-text} % To support LaTeX quoting style

\usepackage[xetex,
    colorlinks,
    citecolor=black,
    filecolor=black,
    linkcolor=black,
    urlcolor=black,
	plainpages=false,
  	pdfpagelabels,
  	bookmarksnumbered,
  	pdftitle={\mytitle},
  	pagebackref,
  	pdfauthor={\myauthor},
  	pdfkeywords={\mykeywords}
  	]{hyperref}

\hypersetup{
    pdfborder = {0 0 0}
}

\setcounter{secnumdepth}{1}
\setcounter{tocdepth}{1}

\begin{document}

\setmainfont
[ Path = fonts/,
UprightFont = Vollkorn-Regular.otf,
ItalicFont = Vollkorn-Italic.otf,
BoldFont = Vollkorn-Bold.otf,
BoldItalicFont = Vollkorn-BoldItalic.otf,
] {Vollkorn}

\setsansfont
[ Path = fonts/,
UprightFont = DejaVuSans.ttf,
ItalicFont = DejaVuSans-Oblique.ttf,
BoldFont = DejaVuSans-Bold.ttf,
BoldItalicFont = DejaVuSans-BoldOblique.ttf,
] {DejaVu Sans Mono}

\setmonofont
[ Path = fonts/,
UprightFont = DejaVuSansMono.ttf,
ItalicFont = DejaVuSansMono-Oblique.ttf,
BoldFont = DejaVuSansMono-Bold.ttf,
BoldItalicFont = DejaVuSansMono-BoldOblique.ttf,
] {DejaVu Sans Mono}

\frontmatter
\setcounter{page}{0}
\pagenumbering{roman}


\begin{titlepage}

\newgeometry{top=.1cm}

\vspace*{15mm}


\begin{center}

\uppercase{\fontsize{28}{28}\sf\bfseries A Comprehensive\\[.3in] Guide to}\\[.3in]
\uppercase{\fontsize{49}{35}\sf\bfseries Python}\\[.3in]
\uppercase{\fontsize{42}{35}\sf\bfseries Programming}\\[.3in]


{\fontsize{21}{21}\sf\bfseries First Edition}\\[.7in]

{\fontsize{28}{28}\sf\bfseries Baiju Muthukadan}


\end{center}

\end{titlepage}


\thispagestyle{plain}
\newpage

{\sf\bfseries A Comprehensive Guide to Python Programming }\\
{\sf First Edition}\\
\textit{(English)}\\[3mm]
{\sf\bfseries Baiju Muthukadan}\\[1in]

\copyright 2014 Baiju Muthukadan\\

The text in this book excluding source code is licensed under the
Creative Commons Attribution-NonCommercial-ShareAlike 4.0
International License:\\
\url{http://creativecommons.org/licenses/by-nc-sa/4.0/legalcode}

The source code in this book is under public domain.


\newpage
\thispagestyle{empty} %% Remove header and footer.
\vspace*{2in}
\begin{center}
{\it To,\\[.07in] My Mother}
\end{center}

\newpage
\thispagestyle{plain}

\tableofcontents

\newpage
\thispagestyle{plain}

\cleardoublepage
\phantomsection
\addcontentsline{toc}{subsection}{Preface}
\chapter*{Preface}

Writing a preface before completing the work is not a good idea.  So I
defer writing the real preface.  I will rewrite this preface once the
book reach a good shape with solid content.

I welcome your corrections and feedback now!  Please have a look at
the \LaTeX{} source here: \url{https://github.com/baijum/pyguide}.  You
can also use the issue tracker given there to submit suggestions or
corrections.

As you can see from the copyright page, the text in this book
excluding source code is licensed under the Creative Commons
Attribution-NonCommercial-ShareAlike 4.0 International License.  And
the source code in this book is under public domain.

\vspace*{.5in}
Baiju Muthukadan\\
Bangalore, India\\
August, 2014


\newpage
\thispagestyle{plain}

\mainmatter
\setcounter{page}{1}
\pagenumbering{arabic}
\cleardoublepage
\pagenumbering{arabic}
\phantomsection
\addcontentsline{toc}{section}{Introduction}
\chapter*{Introduction}
\setcounter{page}{1}
This is to test
\cleardoublepage
\phantomsection
\addcontentsline{toc}{section}{Fundamentals}
\chapter*{Fundamentals}

This is to test
\cleardoublepage
\phantomsection
\addcontentsline{toc}{section}{Data Types}
\chapter*{Data Types}

This is to test
% \input{./conditions.tex}
% \input{./functions.tex}
% \input{./loops.tex}
% \input{./moredatatypes.tex}
% \input{./classes.tex}
% \input{./morefunctions.tex}
% \input{./modules.tex}
% \input{./strings.tex}
% \input{./exceptions.tex}
% \input{./io.tex}

\appendix

\backmatter
\end{document}