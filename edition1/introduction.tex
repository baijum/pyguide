\cleardoublepage
\phantomsection
\chapter{Introduction}

\section{Overview}

Python is a general purpose programming language.  Python can be used
to create a solution for any common programming task.  It is useful
for both non-professional and professional programmers.

This book is prepared for beginners who want to learn programming.
The readers are expected to have a basic knowledge of computers.  This
book will cover all the major topics in Python programming language.

The first 7 chapters are strongly reccommended to read in order.  The
remaining chapers can be read in any order.  This chapter will provide
an introduction to the laguage, brief history and walk through of
remaing chapters.  Then, few suggestions for learning Python using
this book is given towards the end of this chapter.

\section{Python language}

\section{Brief history}

In the beginning of Jan 2009, Guido van Rossum wrote in his blog: "I
started the design and implementation of the language on a cold
Christmas break in Amsterdam, in late December 1989.  It started out
as a typical hobby project.  Little did I know where it would all
lead."

In February 1991, Guido van Rossum published the code (labeled version
0.9.0) to alt.sources Usenet group.

\section{Organization of chapters}

The rest of the book is organized into the following chapters.

\begin{description}
\item[Chapter 2: Basics] \hfill \\
This chapter cover the basics of the Python programming language.
\item[Chapter 3: Data Types] \hfill \\
This chapter cover the fundametal data types in Python.
\end{description}

\section{Setting up working environment}

This section will explain setting up a working environment for your
practice.  You can get assistance from your friend, neighbor or
teacher to setup the environment as given below.

We are going to use Python 3 for practicing.

\begin{center}\begin{sffamily}

\fbox{\parbox{\admonitionwidth}{ \textbf{\large Selecting a Text
          Editor}

\vspace{2mm}

To write code you need a plain text editor.  If you haven't picked a
good editor, better to go for a simple text editor for now.  Something
like GEdit (Windows and GNU/Linux) or Notepad++ (Windows only) would
be a good choice.  If you are already fimiliar with a simple editor,
you may look into an advanced editor like Vim or Emacs.  A good editor
will imp

}}
\end{sffamily}
\end{center}

\subsection{Fedora 20 or above, CentOS 7.x, RHEL 7.x}

Recent versions of Fedora, CentOS and RHEL has Python 3 available for
installation.  You need to perform the following command as a root user.

\begin{verbatim}
# yum install python3
\end{verbatim}

\subsection{Windows}

Windows MSI packages are available from Python website.

\url{http://www.python.org/download}

\section{Suggestions to use this book}
