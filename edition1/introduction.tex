\cleardoublepage
\phantomsection
\chapter{Introduction}

\section{Overview}

Python is a general purpose programming language.  Python can be used
to create a solution for any common programming task.  It is useful
for both non-professional and professional programmers.

This book is prepared for beginners who want to learn programming.
The readers are expected to have a basic knowledge of computers.  This
book will cover all the major topics in Python programming language.

The first 7 chapters are strongly reccommended to read in order.  The
remaining chapers can be read in any order.  This chapter will provide
an introduction to the laguage, brief history and walk through of
remaing chapters.  Then, few suggestions for learning Python using
this book is given towards the end of this chapter.

\section{Python language}

\section{Brief history}

\section{Walk through of chapters}

\section{Setting up working environment}

This section will explain setting up a working environment for your
practice.  You can get assistance from your friend, neighbor or
teacher to setup this.  You must be comfortable using a text editor to
write code.  The text editor could be anything that you already know.
Here are few commonly used editors: GEdit, Notepad++, Vim, Emacs etc.

We are going to use Python 3 for practicing.

\subsection{Fedora 20 or above, CentOS 7.x, RHEL 7.x}

Recent versions of Fedora, CentOS and RHEL has Python 3 available for
installation.  You need to perform the following command as a root user.

\begin{verbatim}
# yum install python3
\end{verbatim}

\subsection{Windows}

Windows MSI packages are available from Python website.

\url{http://www.python.org/download}

\section{Suggestions to use this book}
